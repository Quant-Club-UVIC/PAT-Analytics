\documentclass[12pt]{article}
\usepackage[a4paper, margin=1in]{geometry}
\usepackage{amsmath, amssymb, amsthm}
\usepackage{hyperref}
\usepackage{setspace}
\usepackage{mathtools}

\onehalfspacing

\title{PAT-Analytics Documentation}
\author{}
\date{}

\begin{document}

\maketitle

\tableofcontents
\newpage

\section{Basics}

\subsection{How to Upload this Document}
GitHub does not natively support KaTeX, so we cannot directly render this Markdown file inline.  
Instead, render it locally with \LaTeX{} and update the resulting PDF.

\subsection{Notation}

\begin{equation}
X := \{\text{all tradable instruments (universe)}\}
\label{eq:universe}
\end{equation}

\subsection{Setting Things Up}
Define
\begin{equation}
X = \{ \text{all positions (tradable instruments)} \}
\end{equation}
We have a portfolio at time $ t $, denoted $ P_t \subseteq X $, the set of positions held at time $ t $. Thus,
\begin{equation}
P : \mathbb{R}_{\geq 0} \to \mathbb{P}(X), \quad t \mapsto P_t
\label{eq:portfolio_function}
\end{equation}

The number of \textbf{shares} of instrument $ x \in X $ at time $ t $ is
\begin{equation}
q(x,t) \in \mathbb{R}_{\geq 0}
\label{eq:quantity}
\end{equation}
and the portfolio composition can be defined as
\begin{equation}
P_t = \{x \in X \mid q(x,t) \neq 0\}.
\label{eq:positions_nonzero}
\end{equation}

The \textbf{price} of instrument $ x $ at time $ t $:
\begin{equation}
p(x,t) \in \mathbb{R}_{\geq 0}.
\label{eq:price}
\end{equation}

The \textbf{value} of an instrument $ x $ at time $ t $:
\begin{equation}
V(x,t) = p(x,t) q(x,t)
\label{eq:instrument_value}
\end{equation}

The \textbf{total portfolio market value}:
\begin{equation}
V_p(t) = \sum_{i \in X} V(i,t)
\label{eq:portfolio_value}
\end{equation}

The \textbf{weight} of instrument $ x \in X $ at time $ t $:
\begin{equation}
w(x,t) = \frac{V(x,t)}{\sum_{i \in P_t} V(i,t)} \in \mathbb{R}_{\geq 0}
\label{eq:weight}
\end{equation}

The \textbf{gross return} of instrument $ x $ from $ t $ to $ t + \Delta t $:
\begin{equation}
R(x,t) = \frac{p(x,t + \Delta t)}{p(x, t)} \in \mathbb{R}_{\geq 0}
\label{eq:gross_return}
\end{equation}

The \textbf{net return}:
\begin{equation}
r(x,t) = \frac{p(x,t + \Delta t) - p(x,t)}{p(x, t)} = R(x,t) - 1
\label{eq:net_return}
\end{equation}

We primarily work with $ w(x,t) $ and $ R(x,t) $.

\subsection{No Quantity}
If the user provides only starting weights but no quantities:
\begin{align}
w(x, t_0) &= \frac{p(x, t_0)q(x, t_0)}{V_p(t_0)} \\
q(x, t_0) &= \frac{w(x,t_0)  V_p(t_0)}{p(x, t_0)} 
\end{align}
Thus, we can assign $ V_p(t_0) $ an arbitrary constant value, and solve!

\subsection{Floating Weights}
Users may specify desired weights $ w_0 : X \to \mathbb{R}_{\geq 0} $ and rebalancing frequency $ \delta \in \mathbb{R}_{\geq 0} $.  
From the definition of returns:
\begin{equation}
p(x, t + \Delta t) = p(x,t)R(x,t)
\label{eq:price_evolution}
\end{equation}
Let $ \Delta q_x = q(x, t + \Delta t) - q(x,t) $. Then:
\begin{align}
V(x,t + \Delta t)
&= p(x,t + \Delta t)q(x,t + \Delta t) \nonumber\\
&= [p(x,t)R(x,t)][q(x,t) + \Delta q_x] \nonumber\\
&= p(x,t)R(x,t)q(x,t) + p(x,t)R(x,t)\Delta q_x \nonumber\\
&= V(x,t)R(x,t) + p(x,t)R(x,t)\Delta q_x
\label{eq:value_evolution}
\end{align}

\subsubsection{Case 1: No Rebalancing}
If no rebalancing occurs ($ \Delta q_x = 0 $):
\begin{align}
w(x, t + \Delta t)
&= \frac{V(x,t + \Delta t)}{\sum_{i \in X} V(i, t + \Delta t)} \nonumber\\
&= \frac{V(x,t)R(x,t)}{\sum_{i \in X} V(i,t)R(i,t)} \nonumber\\
&= \frac{w(x,t)R(x,t)}{\sum_{i \in X} w(i,t) R(i,t)}
\label{eq:weight_evolution}
\end{align}
Hence:
\begin{equation}
\boxed{w(x, t + \Delta t) = \frac{w(x,t)R(x,t)}{\sum_{i \in X} w(i,t) R(i,t)}}
\end{equation}

\subsubsection{Case 2: Rebalancing}
If rebalancing occurs ($ t + \Delta t \equiv 0 \pmod{\delta} $), we adjust positions:
\begin{align}
V_p^{pre}(t + \Delta t) &= \sum_{i \in P_t} p(i, t + \Delta t)q(i, t) \label{eq:vpre}\\
V_p^{post}(t + \Delta t) &= C + \sum_{i \in P_t} p(i, t + \Delta t)q(i, t + \Delta t) - F(\Delta q_i)
\label{eq:vpost}
\end{align}
Thus:
\begin{equation}
V_p^{post} = V_p^{pre} + C - \sum_{i \in P_t}F(\Delta q_i)
\label{eq:vpost_balance}
\end{equation}
After trading:
\begin{equation}
V(x,t + \Delta t)^{post} = w_0(x) V_p^{post}(t + \Delta t), \quad \forall x \in P_{t + \Delta t}
\label{eq:post_value}
\end{equation}
Given $ V(x, t + \Delta t)^{post} = p(x, t + \Delta t)q(x, t + \Delta t) $, we get:
\begin{align}
q(x, t + \Delta t) &=  \frac{w_0(x) V_p^{post}(t + \Delta t)}{p(x, t + \Delta t)} \label{eq:q_post}\\
\Delta q_x &=  \frac{w_0(x) V_p^{post}(t + \Delta t)}{p(x, t + \Delta t)} - q(x, t) \label{eq:dq_general}\\
\Delta q_x &=  \frac{w_0(x)[V_p^{pre} + C - \sum_{i \in P_t}F(\Delta q_i)]}{p(x, t + \Delta t)} - q(x, t)
\label{eq:dq_implicit}
\end{align}
Thus:
\begin{equation}
\boxed{\Delta q_x =  \frac{w_0(x)[V_p^{pre} + C - \sum_{i \in P_t}F(\Delta q_i)]}{p(x, t + \Delta t)} - q(x, t)}
\label{eq:dq_box}
\end{equation}
Since $ F $ may depend on $ \Delta q_i $, numerical methods may be necessary.

\paragraph{Flat Fee}
Let $ F(\Delta q) = F_0 $ (constant):
\begin{align}
V_p^{post} &= V_p^{pre} + C - F_0 n \label{eq:flat_fee_vp}\\
\Delta q_x &= \frac{w_0(x) [V_p^{pre} + C - F_0 n]}{p(x, t + \Delta t)} - q(x,t)
\label{eq:flat_fee_dq}
\end{align}
where $ n = |\{x \in X : q(x,t) \neq q(x,t + \Delta t)\}| $.
Note that if there is no commission fee then we simply set $F_0 = 0$.
\paragraph{Proportional Fee}
Let $ F(\Delta q_x) = \gamma |p(x, t + \Delta t) \Delta q_x| $, where $ \gamma $ is the commission rate:
\begin{align}
V_p^{post} &= V_p^{pre} + C - \sum_{i \in P_t}\gamma |p(x,t + \Delta t) \Delta q_i| \label{eq:prop_fee_vp}\\
\Delta q_x &= \frac{w_0(x) [V_p^{pre} + C - \sum_{i \in P_t}\gamma |p(x,t + \Delta t) \Delta q_i|]}{p(x, t + \Delta t)} - q(x,t)
\label{eq:prop_fee_dq}
\end{align}
Now we have an implicit equation in $\Delta q_x$. First for the sake of brevity, let 
$p(x , t + \Delta t) := p$, $w_{0}(x) = w_{0, x}$. Start of with \ref{eq:prop_fee_dq}

\begin{aligned}
        \Delta q_x &= \dfrac{w_{0,x} [z]}{den}
    
\end{aligned}

\end{document}
